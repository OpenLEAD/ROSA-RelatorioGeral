%%******************************************************************************
%%
%% Metodologia.tex
%%
%%******************************************************************************
%%
%% Title......: Introduction
%%
%% Author.....: GSCAR-DFKI
%%
%% Started....: Nov 2013
%%
%% Emails.....: elael@poli.ufrj.br
%%
%% Address....: Universidade Federal do Rio de Janeiro
%%              Caixa Postal 68.504, CEP: 21.945-970
%%              Rio de Janeiro, RJ - Brasil.
%%
%%******************************************************************************


%%******************************************************************************
%% SECTION - Metodologia
%%******************************************************************************

\section{Metodologia}

Com o objetivo de alcan�armos um conceito s�lido foi feita uma pesquisa
\textbf{Pesquisa Bibliogr�fica} (se��o \ref{pesqbib}) foi realizada tendo como
direcionamento uma idealiza��o advinda de uma realiza��o de \emph{brainstorm}
que  contou com toda a experi�ncia dos engenheiros da DFKI.

A partir do resultado dessa pesquisa foi desenvolvido um conceito base, sobre o
qual foram realizadas \textbf{Pesquisas T�cnicas e de Fornecedores} (se��o
\ref{pesqtec}) de forma recursiva e convergente com rela��o aos resultados. Isto
�, com base nas pesquisas t�nicas buscam fornecedores compat�veis e com o
resultado e informa��o dos produtos dos fornecedores encontrados faz-se
novamente uma presquisa t�cnica, agora mais aprofundada, e assim sucessivamente
at� encontrar-se um resultado final satisfat�rio.

Essas  \textbf{Pesquisas T�cnicas e de Fornecedores} j� s�o focadas nos
componentes a serem utilizados, dessa maneira, os fornecedores escolhidos eram
baseados n�o somente na conformidade t�cnica, mas tamb�m tempo de entrega,
dificuldade de importa��o, suporte e reconhecimento.



