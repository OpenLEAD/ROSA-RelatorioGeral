%%******************************************************************************
%%
%% Metodologia.tex
%%
%%******************************************************************************
%%
%% Title......: Introduction
%%
%% Author.....: GSCAR-DFKI
%%
%% Started....: Nov 2013
%%
%% Emails.....: elael@poli.ufrj.br
%%
%% Address....: Universidade Federal do Rio de Janeiro
%%              Caixa Postal 68.504, CEP: 21.945-970
%%              Rio de Janeiro, RJ - Brasil.
%%
%%******************************************************************************


%%******************************************************************************
%% SECTION - Metodologia
%%******************************************************************************

\section{Metodologia}

Após a análise e compreensão do problema de inserção e remoção de Stoplogs, descrito na seção \textbf{Descrição do Problema}, foi feita uma pesquisa
\textbf{Pesquisa Bibliográfica} e \emph{brainstorm} com o objetivo de alcançar um conceito sólido de solução ao problema. 

A partir do resultado dessa pesquisa foi desenvolvido um conceito base de solução robótica, descrito na seção {\bf Escopo}. Baseado neste conceito, foram realizadas pesquisas de tecnologias e de fornecedores (secção  {\bf Pesquisa Tecnológica}) de forma recursiva e convergente com relação aos resultados. Isto
é, com base nas pesquisas de solução tecnológicas possíveis, buscam-se fornecedores compatíveis e com o
resultado e informação dos produtos dos fornecedores encontrados faz-se
novamente uma presquisa de tecnologia	, agora mais aprofundada, e assim sucessivamente
até encontrar-se um resultado final satisfatório. 

Esta pesquisa já é focada nos
componentes a serem utilizados, dessa maneira, os fornecedores escolhidos eram
baseados não somente na conformidade técnica, mas também tempo de entrega,
dificuldade de importação, suporte e reconhecimento. O escopo inicial de solução
é então atualiziado e detalhado de acordo com o resultado desta pesquisa, resultando na descrição do robô a ser construído no projeto (secção {\bf Conclusão do Projeto Básico}).



