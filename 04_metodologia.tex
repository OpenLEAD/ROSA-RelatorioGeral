%%******************************************************************************
%%
%% Metodologia.tex
%%
%%******************************************************************************
%%
%% Title......: Introduction
%%
%% Author.....: GSCAR-DFKI
%%
%% Started....: Nov 2013
%%
%% Emails.....: elael@poli.ufrj.br
%%
%% Address....: Universidade Federal do Rio de Janeiro
%%              Caixa Postal 68.504, CEP: 21.945-970
%%              Rio de Janeiro, RJ - Brasil.
%%
%%******************************************************************************


%%******************************************************************************
%% SECTION - Metodologia
%%******************************************************************************

\section{Metodologia}

Com o objetivo de alcançarmos um conceito sólido foi feita uma pesquisa \textbf{Pesquisa
Bibliográfica} foi realizada tendo como direcionamento uma idealização advinda
de uma realização de \emph{brainstorm} que  contou com toda a experiência dos
engenheiros da DFKI.

A partir do resultado dessa pesquisa foi desenvolvido um conceito base, sobre o
qual foram realizadas \textbf{Pesquisas Técnicas e de Fornecedores} de forma
recursiva e convergente com relação aos resultados. Isto é, com base nas
pesquisas ténicas buscam fornecedores compatíveis e com o resultado e
informação dos produtos dos fornecedores encontrados faz-se novamente uma
presquisa técnica, agora mais aprofundada, e assim sucessivamente até
encontrar-se um resultado final satisfatório.

Essas  \textbf{Pesquisas Técnicas e de Fornecedores} já são focadas nos
componentes a serem utilizados, dessa maneira, os fornecedores escolhidos eram
baseados não somente na conformidade técnica, mas também tempo de entrega,
dificuldade de importação, suporte e reconhecimento.



