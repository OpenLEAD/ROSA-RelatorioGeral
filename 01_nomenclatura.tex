%%******************************************************************************
%%
%% nomenclatura.tex
%%
%%******************************************************************************
%%
%% Title......: Nomenclatura
%%
%% Author.....: GSCAR-DFKI
%%
%% Started....: Nov 2013
%%
%% Emails.....: renan028@gmail.com
%%
%% Address....: Universidade Federal do Rio de Janeiro
%%              Caixa Postal 68.504, CEP: 21.945-970
%%              Rio de Janeiro, RJ - Brasil.
%%
%%******************************************************************************


%%******************************************************************************
%% CHAPTER - Nomenclatura
%%******************************************************************************
\section{Nomenclatura}
\begin{itemize}

\item \emph{Stoplog}: Bloco de a�o com vinte metros de comprimento, tr�s metros de altura e tr�s metros de largura (20x3x3 m). O fluxo de �gua do rio � controlado pelo empilhamento de \emph{Stoplogs} (FIGURA).


\item \emph{Lifting Beam}: Estrutura mec�nica respons�vel pelo deslocamento de \emph{Stoplogs}, composta por: duas garras n�o atuadas, duas chaves de opera��o, vigas e mecanismo. Um guindaste atua neste mecanismo (FIGURA).
    
\item \emph{Guindaste}: O guindaste � capaz de sustentar todo o conjunto \emph{Lifting Beam}/\emph{Stoplog} e � atuado por um motor el�trico.  

\item \emph{Garra pescadora}: Garra localizada no \emph{Lifting Beam} que se prende ao \emph{Stoplog}. O mecanismo � composto por duas garras (FIGURA).

\item \emph{Chave de opera��o}: Localizada na viga principal, pr�xima � garra pescadora, seleciona o modo de opera��o.  Atuada manualmente (FIGURA).

\end{itemize}
