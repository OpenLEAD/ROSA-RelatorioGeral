%%******************************************************************************
%%
%% introduction.tex
%%
%%******************************************************************************
%%
%% Title......: Introduction
%%
%% Author.....: GSCAR-DFKI
%%
%% Started....: Nov 2013
%%
%% Emails.....: renan028@gmail.com
%%
%% Address....: Universidade Federal do Rio de Janeiro
%%              Caixa Postal 68.504, CEP: 21.945-970
%%              Rio de Janeiro, RJ - Brasil.
%%
%%******************************************************************************


%%******************************************************************************
%% SECTION - Introdução
%%******************************************************************************

\section{Introdução}
No primeiro quadrimestre do projeto ROSA, período de 8.10.2013 ao 8.01.2014, foi realizada o projeto básico. O projeto básico é o conjunto de elementos necessários e suficientes que caracterizam o trabalho a ser realizado e resultado de estudos técnicos preliminares que asseguram a viabilidade do projeto. 

Este documento apresentará as etapas realizadas para o desenvolvimento do projeto básico, organizado nas seguintes seções:
    \begin{itemize}
        \item \bf Descrição do problema: descrição do problema e os modos de operação do processo atual.
	\item Metodologia: a técnica de análise ciêntífica utilizada no desenvolvimento do projeto Básico.
        \item Pesquisa Bibliográfica: revisão das contribuições existentes na literatura sobre o problema de operação de Stoplogs alagados.  
	\item Escopo: descrição das características e funções que caracterizam o robô para operação de Stoplogs alagados - ROSA. 
        \item Pesquisa Tecnológica: pesquisa prescritiva aplicada ao escopo do projeto. 
        \item Conclusão: descrição da solução robótica ROSA. 
         \item Revisão do cronograma físico financeiro 
        \item Referências bibliográficas.
        \item Anexo I: Atas de reuniões. 
        \item Anexo II: Especificação dos equipamentos permanetes selecionados (listagem, pinagem e data sheets) 
        \item Documento Complementar I: proposta de tese de mestrado do aluno André Figueiró, bolsista de mestrado. 

    \end{itemize}