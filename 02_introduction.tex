%%******************************************************************************
%%
%% introduction.tex
%%
%%******************************************************************************
%%
%% Title......: Introduction
%%
%% Author.....: GSCAR-DFKI
%%
%% Started....: Nov 2013
%%
%% Emails.....: renan028@gmail.com
%%
%% Address....: Universidade Federal do Rio de Janeiro
%%              Caixa Postal 68.504, CEP: 21.945-970
%%              Rio de Janeiro, RJ - Brasil.
%%
%%******************************************************************************


%%******************************************************************************
%% SECTION - Introdṳ̣o
%%******************************************************************************

\section{Introdṳ̣o}
O objetivo deste relat�?rio ̩ apresentar o conceito do projeto ROSA. Este documento apresentar�� os modos de opera̤̣o do sistema atual e suas falhas, descrever�� o sistema proposto, as poss�?veis solṳ̵es e a metodologia de pesquisa para o desenvolvimento de solṳ̵es. Por ̼ltimo, este relat�?rio mostrar�� um fluxograma completo da opera̤̣o com o novo sistema.

Este relat�?rio se organiza nas seguintes se̵̤es:
    \begin{itemize}
        \item Descri̤̣o do problema: descri̤̣o do problema e os modos de opera̤̣o do processo atual.
        \item Sistema proposto: especifica̤̣o do sistema proposto, como sensores e atuadores que seṛo utilizados para a concluṣo dos objetivos do projeto.
        \item Metodologia: descri̤̣o do m̩todo de pesquisa e desenvolvimento de solṳ̵es do grupo.
        \item Pesquisa bibliogr��fica: levantamento das tecnologias utilizadas no sistema proposto e as tecnologias estudadas durante a fase de desenvolvimento do sistema.
        \item Fluxograma da solṳ̣o: diagrama l�?gico com as etapas do processo, falhas e solṳ̵es.
        \item Refer̻ncias.
    \end{itemize}