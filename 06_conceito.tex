% %******************************************************************************
% % % conceito.tex %
% %******************************************************************************
% % % Title......: Projeto Conceitual % % Author.....: GSCAR-DFKI % %
% Started....: Nov 2013 % % Emails.....: renan028@gmail.com % % Address....:
% Universidade Federal do Rio de Janeiro %              Caixa Postal 68.504,
% CEP: 21.945-970 %              Rio de Janeiro, RJ - Brasil.
% %
% %******************************************************************************


% %******************************************************************************
% % SECTION - Sistema proposto
% %******************************************************************************\


\chapter{Projeto Conceitual}
Esta seção aborda a solucão conceitual construída a partir da análise dos modos de operação e 
falhas expostos na seção {\bf Descrição do Problema} e da revisão bibliográfica realizada na 
seção {\bf Revisão Bibliográfica}. Conceitualmente, o robô ROSA será constituído por um conjunto de sensores e
atuadores a prova d'água que serão instalados no \emph{Lifting Beam}.
As subseções que se seguem descrevem o projeto conceitual direcionado a cada
falha de operação, definindo assim o sensoriamento e atuação necessária para o sistema. 


%%%%%%%%%%%%%%%%%%

\section{Operação excepcional de inserção 1 - Travamento durante inserção}

As falhas consequentes da operação exceptional de inserção 1 - travamento do
\emph{stoplog}, ocorrem devido à inclinação excessiva do \emph{stoplog} no
trilho e pela liberação prematura do \emph{stoplog}. Logo, os mesmos podem ser evitados através do monitoramento da inclinação do \emph{Lifting Beam} e pelo monitoramento do engate entre \emph{Garra Pescadora} e  \emph{stoplog}.

Dado o acoplamento mecânico entre o \emph{Lifting Beam} e o \emph{stoplog} é
possível medir a inclinação do \emph{stoplog} através da medição da inclinação do \emph{Lifting Beam}. Logo, a solução conceitual para executar este monitoramento será a instalação de um inclinômetro dentro da eletrônica embarcada acoplada ao \emph{Lifting Beam}.

Devido à geometria do \emph{Lifiting Beam}, no
desencaixe entre \emph{Garra Pescadora} e \emph{Stoplogs}, a \emph{Garra
Pescadora} deve obedecer uma sequência de posições angulares conhecidas. A
partir desse fato, é possível monitorar o status do engate por meio de monitoramento da posição da \emph{Garras Pescadoras} com relação ao \emph{Lifiting Beam}.
Cada \emph{Garra Pescadora} é acoplada ao \emph{Lifiting Beam} através de um único eixo, logo a posição da mesma é diretamente relacionada a posição angular deste eixo. Por conseguinte, através da medição da posição angular deste eixo é possível medir a posição da \emph{Garra Pescadora}. 

O engate entre a \emph{Garras Pescadoras} e o \emph{Stoplog} pode ser medido
também através de sensores de contatos nas extremidades das \emph{Garras Pescadoras}.
Considerando o fato que não é possível prever todos os cenários que podem
ocorrer durante a operação do \emph{Stoplog}, a solução conceitual irá considerar ambas
as soluções para medição do engate/desengate.

Vale ressaltar que o projeto conceitual visa o monitoramento da operações. O
operador, a partir dos dados recebidos, pode decidir em continuar a tarefa ou
reiniciá-la, podendo assim evitar as situações extremas que resultariam em danos a infraestrutura. 

%%%%%%%%%%%%%%%%%%

\section{Operação excepcional de inserção 2 - Falha do desencaixe da garra pescadora}
\label{op:sol:ins:1}

As consequências danosas de um desencaixe mal sucedido entre o \emph{Stoplog} e
as \emph{Garras Pescadoras}, como desencaixe parcial, ocorrem
principalmente devido à falta de feedback na operação de inserção dos \emph{Stoplogs}.
Logo, a solução conceitual será o monitoramento redundante (medição da posição
angular e do contato) do status do engate e por conseguinte evitar possíveis danos à estrutura.

A solução concebida não atua diretamente na movimentação das garras, entretanto
possibilita um monitoramento de todos os parâmetros fundamentais para uma operação correta e eficiente e, assim, permite, em tempo real, que ajustes sejam realizados para se finalizar a
operação com sucesso. 

A solução da atuação ativa das \emph{Garras Pescadoras} através de motores foi
discutida com os operadores de \emph{Stoplogs} e desconsiderada como uma solução
viável, pois tal solução iria alterar significativamente a estrutura mecânica do sistema e, por conseguinte, reduziria o grau de robustez do mesmo. Logo, a aplicação ou não de tal atuador ficará em aberto para futuras discussões e não será parte da solução conceitual inicial do projeto ROSA.

%%%%%%%%%%%%%%%%%%



\section{Operação excepcional de inserção 3 - Não vedamento devido ao acumulo de detritos na base do trilho}

Atualmente, não existe um método eficiente de se realizar uma inspeção prévia da
base do trilho do \emph{Stoplogs} antes da inserção. A má vedação, em geral, só é
detectada quando o escoamento do circuito hidráulico falha, resultando no envio de mergulhadores para averiguação da causa do problema.

Logo, propõe-se uma solução de inspeção inicial através da realização do mapeamento 3D da base do trilho.
O mapeamento 3D proporciona a vantagem de não apenas detectar a existência,
mas, também, de se conhecer a extensão e volume do detrito/silte acumulado.
Logo, possibilitando uma decisão informada do método mais eficiênte para remoção do detrito/silte.

A solução conceitual para remoção de pequenos detritos e acúmulos de silte será
através de um sistema de bombeamento submarino direcionado a região obstruída. A metodologia/sistema para a remoção de detritos/silts acumulados na base do trilho de grande volume vai além da robotização da \emph{Garra Pescadora} e não faz parte do escopo deste projeto.


%%%%%%%%%%%%%%%%%%

\section{Operação excepcional de remoção 1 - Falha no encaixe}

A solução conceitual de monitoramento redundante (medição da posição angular e
do contato) do status do engate, possibilita verificar se a operação de engate foi realizada com sucesso. Evitando, assim, a tentativa de remoção do \emph{Stoplog} quando o engate for apenas parcial, o que pode vir a danificar a estrutura.

Entretanto, apenas o monitoramento do status do engate não permite ao operador determinar a causa da falha. Logo, a solução conceitual irá incluir um sistema para mapeando 3D do topo do \emph{Stoplog} e da região do olhal, o que possibilita a visualização do problema sem a necessidade do envio
de mergulhadores. 

A causa mais comum, de acordo com o pessoal de operação da ESBR, para este tipo
de falha é o acúmulo de detritos/silte na região do olhal. Logo, o projeto irá apresentar uma solução conceitual de limpeza ativa através de uma bomba submarina. O mapeamento de todas as causas possíveis para a falha do engate e suas soluções vai além do escopo deste projeto.

%%%%%%%%%%%%%%%%%%

\section{Operação excepcional de remoção 2 - Travamento durante remoção}

O travamento do \emph{Stoplog} durante a remoção é causado pela inclinação excessiva do Stoplog no trilho, o que resulta em um desalinhamento e subsequentemente no travamento da operação. Logo, através do monitoramento contínuo por um inclínometro instalado no \emph{Lifting Beam} é possível que o operador tome as medidas preventivas necessárias para evitar o travamento.   

%%%%%%%%%%%%%%%%%%

\section{Operação excepcional de remoção 3 - Acúmulo de sedimentos no fundo}
O método para a resolução da condição de acúmulo de sedimentos no fundo,
descrita na subseção \ref{op:rem:3}, não faz parte do escopo deste projeto.


%%%%%%%%%%%%%%%%%%%%

\section{Conclusão do Conceito Básico}

Dados as falhas e os conceitos desenvolvidos nas subseções acima, o
sensoriamento e atuação necessária para o robô ROSA será:

\begin{itemize}

	\item medição de contato entre a \emph{Garra Pescadora} e o \emph{Stoplog}; 
	\item medição do posicionamento angular da \emph{Garra Pescadora}; 
	\item medição da inclinação do \emph{Lifting Beam};
	\item medição da profundidade do \emph{Lifting Beam}; 
	\item mapeamento 3D do fundo do trilho e do topo do \emph{Stoplog}; e
	\item limpeza por jato de água pressorizado. 

\end{itemize}

Os sensores medirão dados detalhados sobre o atual status da
operação de inserção/remoção dos \emph{stoplogs} permitindo ao operador tomar decisões
com base nessas informações, otimizar a operação e evitar possíveis problemas.
Entretanto, para o mesmo, os dados medidos em baixo d'agua precisam estar
disponíveis para vizualização do operador que se encontra no pórtigo rolante e controla o sistema de inserção e remoção do \emph{Stoplog}.

Logo, os sensores serão conectados a uma eletrônica embarcada a prova d'água,
instalada também no \emph{Lifting Beam}, que pré-processará e transmitirá as
informações para a superfície através de um umbilical. 
O umbilical também será utilizado para transmitir energia para a eletrônica embarcada subaquática e deverá funcionar passivamente junto ao pórtigo rolante através de um sistem de gerência de umbilical. 
Na supérficie, uma eletrônica de terra constítuida por um computador embarcado e um sistema de potência receberá e pós-processará os dados, transmitindo os mesmos via rede sem fio. 
Um table conectado na rede sem fio será instalado na cabine do operador, possibilitando a vizualisação dos dados em uma interface gráfica.

O conceito da estrutura física necessária para o robô ROSA e sua utilização será: 

\begin{itemize}

	\item eletrônica embarcada subaquática; 
	\item umbilical;
	\item sistema de gerência de umbilical;
	\item eletrônica de terra; e
	\item tablet para interface do usuário. 

\end{itemize}

O sistema de limpeza por jato de água pressurizado possibilita intervir na
operação, resolvendo problemas encontrados, sem a necessidade de enviar
mergulhadores ao local. Entretanto, a posição precisa na qual o sistema será
montado dependerá da localização específica da obstrução. Logo, o mesmo será utilizado caso a caso independete da eletrônica embarcada.


