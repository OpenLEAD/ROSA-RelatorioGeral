% %******************************************************************************
% % % conceito.tex %
% %******************************************************************************
% % % Title......: Projeto Conceitual % % Author.....: GSCAR-DFKI % %
% Started....: Nov 2013 % % Emails.....: renan028@gmail.com % % Address....:
% Universidade Federal do Rio de Janeiro %              Caixa Postal 68.504,
% CEP: 21.945-970 %              Rio de Janeiro, RJ - Brasil.
% %
% %******************************************************************************


% %******************************************************************************
% % SECTION - Sistema proposto
% %******************************************************************************
\section{Projeto Conceitual}
Esta seção aborda os problemas que serão atacados pelo projeto, de acordo com os
modos de operação e falhas expostos na seção Descrição do Problema.

Conceitualmente, o robô ROSA será constituído por um conjunto de sensores e
atuadores a prova d'água que serão instalados no \emph{Lifting Beam}. Os
sensores e atuadores serão conectados a uma eletrônica embarcada a prova d'água,
instalada também no \emph{Lifting Beam}, que processará e transmitirá as
informações para a superfície através de um umbilical. Na superfície, os dados e
controles do sistema poderão ser visualizados em uma interface gráfica no
console de comando. Os sensores medirão dados detalhados sobre o atual status da
operação de inserção/remoção dos \emph{stoplogs} permitindo ao operador tomar decisões
com base nessas informações, otimizar a operação e evitar possíveis problemas.
Os atuadores possibilitam intervir na operação resolvendo problemas encontrados
sem a necessidade de enviar mergulhadores ao local.

Os principais requisitos de projeto são robustez dos dispositivos, capacidade de
submersibilidade (IP69K), resistência a choque, vibração, e campos magnéticos
externos não devem afetar as medições.

As subseções que se seguem descrevem o projeto conceitual direcionado a cada
falha de operação.

%%%%%%%%%%%%%%%%%%

\subsection{Operação excepcional de inserção 1 - Travamento durante inserção} O
travamento do \emph{stoplog} durante a inserção pode ser verificado pelo
monitoramento da inclinação do \emph{Lifting Beam}, pela constante verificação
do encaixe a partir do eixo de rotação das garras e pelo acompanhamento da
movimentação do \emph{Lifting Beam} quando submerso.

Sensores de inclinação podem fornecer ao operador a informação de que o
\emph{stoplog} está seguindo ou não o curso do trilho corretamente.

Mesmo em caso de não inclinação, há a possibilidade de o \emph{stoplog} ser
assentado de maneira incorreta. O acúmulo uniforme de sedimentos pode criar uma
camada e impedir o posicionamento correto do \emph{stoplog}. Os sensores de
profundidade possibilitam que o operador monitore a finalização da tarefa.

Sensores de rotação podem ser acoplados ao eixo de rotação das garras
pescadoras e sensores de contato na região do encaixe entre \emph{garras pescadoras} e o \emph{Stoplog}, monitorando constantemente  o status do engate do conjunto.

Vale ressaltar que o projeto conceitual visa o monitoramento da operações. O
operador, a partir dos dados recebidos, pode decider em continuar a tarefa ou
reiniciá-la, podendo assim evitar as situações extremas que resultariam em danos a infraestrutura. 

%%%%%%%%%%%%%%%%%%

\subsection{Operação excepcional de inserção 2 - Falha do desencaixe da garra pescadora}
\label{op:sol:ins:1}

As consequências danosas de um desencaixe mal sucedido entre o \emph{Stoplog} e
as \emph{Garras Pescadoras}, como desencaixe parcial e travamento de
\emph{Stoplogs}, são
principalmente devido à falta de feedback na operação de inserção dos
\emph{Stoplogs}.

Devido à geometria do \emph{Lifiting Beam} e das \emph{Garras Pescadoras}, o
desencaixe da \emph{Garra Pescadora} dos \emph{Stoplogs} tem, necessariamente,
um conjunto de posições e uma ordem de acontecimento dessas posições que a
\emph{Garra Pescadora} deve obedecer. A partir desse fato, é possível monitorar
a operação de desencaixe por meio de sensores de rotação acoplados às
\emph{Garras Pescadoras}, assim como, por sensores de contatos na extremidade da \emph{Garras Pescadoras}. 

Por meio do monitoramento de todas as etapas de movimentação das \emph{Garras
Pescadoras} e o alinhamento do \emph{Lifiting Beam}, o operador tem a capacidade
de perceber que a operação está sendo realizada corretamente e caso algum erro
ocorra, pode decidir entre abortar a operação e reiniciá-la ou, caso seja
necessário, aborta-la completamente e tomar ações corretivas mais drásticas,
como o envio de mergulhadores.

A solução concebida não atua diretamente na movimentação das garras, entretanto possibilita um monitoramento de todos os
parâmetros fundamentais para uma operação correta e eficiente e, assim,  
permite, em tempo real, que ajustes sejam realizados para se finalizar a
operação com sucesso.

%%%%%%%%%%%%%%%%%%

\subsection{Operação excepcional de inserção 3 - Não vedamento devido ao acumulo de detritos na base do trilho}

Atualmente, não existe um método eficiente de se realizar uma inspeção prévia da base do trilho do Stoplogs antes da inserção. A má vedação em geral só é detectada quando o escoamento do circuito hidráulico falha, resultando no envio de mergulhadores para averiguação da causa do problema. 

Logo, propõe-se uma solução de inspeção inicial através do uso
de um sonar que mapeará a base do trilho e possibilitará a
detecção de detritos ou ácumulo de silt que resultariam em uma falha da vedação. 

O mapeamento por sonar também proporciona a vantagem de não apenas se detectar a existência, mas também, de se conhecer a extensão e volume do detrito/silt acumulado. Logo, possibilitando uma decisão informada e consequentemente mais eficiênte da maneira de se remover o detrito/silt. 

%%%%%%%%%%%%%%%%%%

\subsection{Operação excepcional de remoção 1 - Falha no encaixe}

O monitoramento da movimentação da  \emph{Garras Pescadoras} e do contato entre \emph{Garras Pescadoras} e o \emph{Stoplog}., descrito na subseção
\ref{op:sol:ins:1}, possibilita verificar se a operação de engate foi realizada com sucesso. Evitando assim, a tentativa de remoção do \emph{Stoplog} quando o engate for apenas parcial. 


Caso haja a falha no engate é possível que haja alguma obstrução no olhal do
\emph{Stoplog} e/ou em sua superfície. Deve-se, então realizar uma operação de
inspeção por sonar, mapeando o topo do stoplog e a região do olhal, o que possibilita a visualização do problema sem a necessidade do envio
de mergulhadores. Se a obstrução for decorrente de acúmulo de silt sobre o olhal é possível acoplar uma bomba submarina ao \emph{lifting beam} para remoção do material acúmulado. 

%%%%%%%%%%%%%%%%%%

\subsection{Operação excepcional de remoção 2 - Travamento durante remoção}

O travamento do \emph{Stoplog} durante a remoção é causado pela inclinação excessiva do Stoplog no trilho, o que resulta em um desalinhamento e subsequentemente no travamento da operação. Logo, através do monitoramento contínuo por um inclínometro instalado no \emph{Lifting Beam} é possível que o operador tome as medidas preventivas necessárias para evitar o travamento.   

%%%%%%%%%%%%%%%%%%

\subsection{Operação excepcional de remoção 3 - Acúmulo de sedimentos no fundo}
O método para a resolução da condição de acúmulo de sedimentos no fundo,
descrita na subseção \ref{op:rem:3}, não faz parte do escopo deste projeto.
