% %******************************************************************************
% % % conceito.tex %
% %******************************************************************************
% % % Title......: Projeto Conceitual % % Author.....: GSCAR-DFKI % %
% Started....: Nov 2013 % % Emails.....: renan028@gmail.com % % Address....:
% Universidade Federal do Rio de Janeiro %              Caixa Postal 68.504,
% CEP: 21.945-970 %              Rio de Janeiro, RJ - Brasil.
% %
% %******************************************************************************


% %******************************************************************************
% % SECTION - Sistema proposto
% %******************************************************************************
\section{Projeto Conceitual}
Esta seção aborda os problemas que serão atacados pelo projeto, de acordo com os
modos de operação e falhas expostos na seção Descrição do Problema.

Conceitualmente, o robô ROSA será constituído por um conjunto de sensores e
atuadores a prova d'água que serão instalados no \emph{Lifting Beam}. Os
sensores e atuadores serão conectados a uma eletrônica embarcada a prova d'água,
instalada também no \emph{Lifting Beam}, que processará e transmitirá as
informações para a superfície através de um umbilical. Na superfície, os dados e
controles do sistema poderão ser visualizados em uma interface gráfica no
console de comando. Os sensores medirão dados detalhados sobre o atual status da
operação de inserção/remoção dos stoplogs permitindo ao operador tomar decisões
com base nessas informações, otimizar a operação e evitar possíveis problemas.
Os atuadores possibilitam intervir na operação resolvendo problemas encontrados
sem a necessidade de enviar mergulhadores ao local.

A informação de encaixe mal ou bem sucedido entre garra e stoplog é de grande
importância para o operador, durante o processo de remoção e inserção. Caso a
operação continue sendo executada durante um encaixe mal sucedido, pode haver
danos ao  \emph{Lifting Beam} ao stoplog e ao trilho, além de impossibilitar a
finalização da tarefa. Há, portanto, a necessidade de instrumentar a garra
pescadora com sensores que possam fornecer a informação de encaixe bem sucedido,
uma eletrônica embarcada para receber e processar essa informação e um sistema
eletrônico na base que deverá fornecer ao operador o status do processo.
Os principais requisitos de projeto são robustez dos dispositivos, capacidade de
submersibilidade (IP69K), resistência a choque, vibração, e campos magnéticos
externos não devem afetar as medições.

As subseções que se seguem descrevem o projeto conceitual direcionado a cada
falha de operação.

\subsection{Operação excepcional de inserção 1 - Travamento durante inserção} O
travamento do \emph{stoplog} durante a inserção pode ser verificado pelo
monitoramento da inclinação do \emph{Lifting Beam}, pela constante verificação
do encaixe a partir do eixo de rotação das garras e o acompanhamento da
movimentação do \emph{Lifting Beam} quando submerso.

Sensores de inclinação podem fornecer ao operador a informação de que o
\emph{stoplog} está seguindo ou não o curso do trilho corretamente.

Mesmo em caso de não inclinação, há a possibilidade de o \emph{stoplog} ser
assentado de maneira incorreta. O acúmulo uniforme de sedimentos pode criar uma
camada e impedir o posicionamento correto de \emph{stoplog}. Os sensores de
profundidade possibilitam que o operador monitore a finalização da tarefa.

Sensores de rotação podem ser acoplados ao eixo de rotação das garras
pescadoras, monitorando constantemente o encaixe das garras e alertando ao
operador situações de inclinações extremas que estejam desacoplando o conjunto
\emph{Lifting Beam}/\emph{stoplog}.

Vale ressaltar que o projeto conceitual visa o monitoramento da operações. O
operador, a partir dos dados recebidos, pode decider em continuar a tarefa ou
reiniciá-la.

\subsection{Operação excepcional de inserção 2 - Falha do desencaixe da garra pescadora}
\label{op:sol:ins:1}

As consequências danosas de um desencaixe mal sucedido entre o \emph{Stoplog} e
as \emph{Garras Pescadoras}, como desencaixe parcial e travamento de
\emph{Stoplogs}, descritas nas subseções \ref{op:ins:1} e \ref{op:rem:1}, são
principalmente devido à falta de feedback na operação de inserção dos
\emph{Stoplogs}.

Devido à geometria do \emph{Lifiting Beam} e das \emph{Garras Pescadoras}, o
desencaixe da \emph{Garra Pescadora} dos \emph{Stoplogs} tem, necessariamente,
um conjunto de posições e uma ordem de acontecimento dessas posições que a
\emph{Garra Pescadora} deve obedecer. A partir desse fato, é possível monitorar
a operação de desencaixe por meio de sensores de rotação acoplados às
\emph{Garras Pescadoras}.

Por meio do monitoramento de todas as etapas de movimentação das \emph{Garras
Pescadoras} e o alinhamento do \emph{Lifiting Beam}, o operador tem a capacidade
de perceber que a operação está sendo realizada corretamente e caso algum erro
ocorra, pode decidir entre abortar a operação e reiniciá-la ou, caso seja
necessário, aborta-la completamente e tomar ações corretivas mais drásticas,
como o envio de mergulhadores.

A solução concebida não atua diretamente na movimentação das garras e realiza um
desencaixe livre de erros.Entretanto, possibilita um monitoramento de todos os
parâmetros fundamentais para uma operação correta e eficiente e, assim,  
permite, em tempo real, que ajustes sejam realizados para se finalizar a
operação com sucesso ou, se necessário,  a operação seja abortada para evitar
possíveis danos ou envio desnecessário de mergulhadores.

\subsection{Operação excepcional de inserção 3 - Não vedamento devido ao acumulo de detritos na base do trilho}

Como forma de evitar o retrabalho, propõe-se uma inspeção inicial através do uso
de um sonar \emph{profiling} que mapeará o fundo do rio e possibilitará a
identificação de detritos presentes. Estes, caso não tratados, podem causar
os problemas descritos na subseção \ref{op:rem:3}. A proposta de implementação
dessa inspeção é mais profundamente descrita na subseção \ref{sis:sol:1}.

O mapeamento também proporciona uma limpeza dos detritos mais eficiente devido
ao conhecimento da localização do que deve ser removido. Limpeza essa feita
por meio de uma bomba submarina capaz de eficientemente remover pequenos
detritos, entretanto para outros de grande porte (e.g. troncos de árvore
submersos) faz-se necessário o uso de um equipamento do tipo garra e guindaste,
já utilizado atualmente.



\subsection{Operação excepcional de remoção 1 - Falha no encaixe}

O monitoramento da movimentação da garra, descrito na subseção
\ref{op:sol:ins:1}, possibilita de maneira totalmente análoga um monitoramento
da operação de encaixe entre as \emph{Garras Pescadoras} e o \emph{Stoplog}.

Entretanto, se após múltiplas tentativas a operação de encaixe não seja
realizada com sucesso, é possível que haja alguma obstrução no olhal do
\emph{Stoplog} e/ou em sua superfície. Deve-se, então realizar uma operação de
inspeção que possibilite a visualização do problema sem a necessidade do envio
de mergulhadores. Uma vez que a causa do problema e sua localização sejam
identificadas, um sistema de limpeza, controlado remotamente, pode ser enviado.

Para situações como a obstrução do olhal ou pequenos objetos na superfície do
\emph{Stoplog} é suficiente a operação de limpeza, porém, em casos extremos,
pode-se usar uma garra ou guindaste extra para auxiliar a eliminação dos
detritos e, em último caso, a utilização de mergulhadores.


\subsection{Operação excepcional de remoção 2 - Travamento durante remoção}
O procedimento é exatamente como no caso da inserção.

\subsection{Operação excepcional de remoção 3 - Acúmulo de sedimentos no fundo}
O método para a resolução da condição de acúmulo de sedimentos no fundo,
descrita na subseção \ref{op:rem:3}, não faz parte do escopo deste projeto.
