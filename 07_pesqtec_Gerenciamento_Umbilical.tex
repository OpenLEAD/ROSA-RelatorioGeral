\subsection{Sistema de Gerenciamento de Umbilical}

Carretéis industriais são dispositivos para recolhimento de cabos atuados por mola. Os cabos são fixados e conectados em contatos girantes.Tais dispositivos apresentam robustez estrutural e resistência ao tempo.
 
A partir da pesquisa realizada, é possível concluir que há boa disponibilidade de carretéis industriais para a aplicação visada, inclusive com graus de proteção adequados para umidade e resistência ao tempo (NEMA4) e perfil compacto. Vale notar, portanto, necessidades essenciais para a correta especificação e definição do produto:
\begin{itemize}
  \item O local disponível para fixação do carretel, calculando-se o comprimento ativo do cabo (a diferença de comprimento entre o cabo totalmente recolhido e o cabo totalmente não recolhido), o comprimento máximo suspenso do cabo e o comprimento máximo submerso do cabo.
  
  \item O tipo de cabo a ser utilizado, seu peso e resistência à tração, bem como o número de fios e o diâmetro deste. O número e espessura dos fios deve ser considerado não só pelo efeito deste no peso do cabo, mas também devido à necessidade dos contatos girantes para cada condutor
 
  \item A opção pelo uso de fibra ótica elimina a possibilidade de uso dos carretéis analisados, visto que não só há a questão de possíveis danos à fibra devido ao enrolamento do cabo como, principalmente, há a necessidade do uso de um acoplamento ótico girante, que não é disponibilizado nos carretéis analisados.
\end{itemize}
\paragraph{Conclusão de análise técnica}\mbox{}\\
 
Um carretel deve ser utilizado para alimentação da bomba, tendo em vista a operação excepcional. Um segundo carretel pode ser utilizado para alimentação e comunicação com o sistema eletrônico para operação convencional, tendo como alternativa o uso de um sistema de bateria e de comunicação por ultrassom.
 
Os fornecedores analisados que apresentam disponibilidade de carretéis adequados à aplicação visada são Cavotec e Conductix. Ambas as empresas apresentam representação no Brasil.
 