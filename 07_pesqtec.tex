%%******************************************************************************
%%
%% pesqbib.tex
%%
%%******************************************************************************
%%
%% Title......: Introduction
%%
%% Author.....: GSCAR-DFKI
%%
%% Started....: Nov 2013
%%
%% Emails.....: renan028@gmail.com
%%
%% Address....: Universidade Federal do Rio de Janeiro
%%              Caixa Postal 68.504, CEP: 21.945-970
%%              Rio de Janeiro, RJ - Brasil.
%%
%%******************************************************************************


%%******************************************************************************
%% SECTION - Pesquisa Técnica
%%******************************************************************************
\setcounter{secnumdepth}{3}
\section{Pesquisa Tecnológica}
\label{pesqtec}


A pesquisa tecnológica visa definir os componentes, métodos e fornecedores que possibilitam a realização da solução conceiual do robô ROSA. Os principais requisitos de projeto são robustez dos dispositivos, capacidade de submersibilidade (IP69K), resistência a choque, vibração, e campos magnéticos
externos não devem afetar as medições. As subseções que se seguem descrevem a pesquisa tecnologica realizada direcionado a soluções conceitual apresentada na {\bf secção Escopo}, estas são: 

\begin{itemize}

	\item Sensor de Contato; 
	\item Posicao Angular; 
	\item Mapeamento 3D; 
	\item Sistema de Gerência de Umbilical;

\end{itemize}


